\selectlanguage{finnish}%

\section{Symmetries}\label{sec:Symmetries}

If the system has nontrivial point group symmetries, group theory
gives additional understanding of the system properties, and can be
used to reduce the computational costs. 

As an example, if our system has a $D_{2h}$ symmetry and our truncated
$\left(I-T\trops\right)$ matrix has size $N\times N$, it can be
block-diagonalized into eight blocks of size about $N/8\times N/8$,
each of which can be LU-factorised separately (this is due to the
fact that $D_{2h}$ has eight different one-dimensional irreducible
representations). This can reduce both memory and time requirements
to solve the scattering problem (\ref{eq:Multiple-scattering problem block form})
by a factor of 64.

In periodic systems (problems (\ref{eq:Multiple-scattering problem unit cell block form}),
(\ref{eq:lattice mode equation})) due to small number of particles
per unit cell, the costliest part is usually the evaluation of the
lattice sums in the $W\left(\omega,\vect k\right)$ matrix, not the
linear algebra. However, the lattice modes can be searched for in
each irrep separately, and the irrep dimension gives a priori information
about mode degeneracy.

\subsection{Excitation coefficients under point group operations}

In order to use the point group symmetries, we first need to know
how they affect our basis functions, i.e. the VSWFs.

Let $g$ be a member of orthogonal group $O(3)$, i.e. a 3D point
rotation or reflection operation that transforms vectors in $\reals^{3}$
with an orthogonal matrix $R_{g}$:
\[
\vect r\mapsto R_{g}\vect r.
\]
Spherical harmonics $\ush lm$, being a basis the $l$-dimensional
representation of $O(3)$, transform as \cite[???]{dresselhaus_group_2008}
\[
\ush lm\left(R_{g}\uvec r\right)=\sum_{m'=-l}^{l}D_{m,m'}^{l}\left(g\right)\ush l{m'}\left(\uvec r\right)
\]
where $D_{m,m'}^{l}\left(g\right)$ denotes the elements of the \emph{Wigner
matrix} representing the operation $g$. By their definition, vector
spherical harmonics $\vsh 2lm,\vsh 3lm$ transform in the same way,
\begin{align*}
\vsh 2lm\left(R_{g}\uvec r\right) & =\sum_{m'=-l}^{l}D_{m,m'}^{l}\left(g\right)\vsh 2l{m'}\left(\uvec r\right),\\
\vsh 3lm\left(R_{g}\uvec r\right) & =\sum_{m'=-l}^{l}D_{m,m'}^{l}\left(g\right)\vsh 3l{m'}\left(\uvec r\right),
\end{align*}
but the remaining set $\vsh 1lm$ transforms differently due to their
pseudovector nature stemming from the cross product in their definition:
\[
\vsh 3lm\left(R_{g}\uvec r\right)=\sum_{m'=-l}^{l}\widetilde{D_{m,m'}^{l}}\left(g\right)\vsh 3l{m'}\left(\uvec r\right),
\]
where $\widetilde{D_{m,m'}^{l}}\left(g\right)=D_{m,m'}^{l}\left(g\right)$
if $g$ is a proper rotation, but for spatial inversion operation
$i:\vect r\mapsto-\vect r$ we have $\widetilde{D_{m,m'}^{l}}\left(i\right)=\left(-1\right)^{l+m}D_{m,m'}^{l}\left(i\right)$.
The transformation behaviour of vector spherical harmonics directly
propagates to the spherical vector waves, cf. (\ref{eq:VSWF regular}),
(\ref{eq:VSWF outgoing}):
\begin{align*}
\vswfouttlm 1lm\left(R_{g}\vect r\right) & =\sum_{m'=-l}^{l}\widetilde{D_{m,m'}^{l}}\left(g\right)\vswfouttlm 1l{m'}\left(\vect r\right),\\
\vswfouttlm 2lm\left(R_{g}\vect r\right) & =\sum_{m'=-l}^{l}D_{m,m'}^{l}\left(g\right)\vswfouttlm 2l{m'}\left(\vect r\right),
\end{align*}
(and analogously for the regular waves $\vswfrtlm{\tau}lm$).  For
convenience, we introduce the symbol $D_{m,m'}^{\tau l}$ that describes
the transformation of both types (``magnetic'' and ``electric'')
of waves at once:
\[
\vswfouttlm{\tau}lm\left(R_{g}\vect r\right)=\sum_{m'=-l}^{l}D_{m,m'}^{\tau l}\left(g\right)\vswfouttlm{\tau}l{m'}\left(\vect r\right).
\]
Using these, we can express the VSWF expansion (\ref{eq:E field expansion})
of the electric field around origin in a rotated/reflected system,
\[
\vect E\left(\omega,R_{g}\vect r\right)=\sum_{\tau=1,2}\sum_{l=1}^{\infty}\sum_{m=-l}^{+l}\sum_{m'=-l}^{l}\left(\rcoefftlm{\tau}lmD_{m,m'}^{\tau l}\left(g\right)\vswfrtlm{\tau}l{m'}\left(k\vect r\right)+\outcoefftlm{\tau}lmD_{m,m'}^{\tau l}\left(g\right)\vswfouttlm{\tau}l{m'}\left(k\vect r\right)\right),
\]
which, together with the $T$-matrix definition, (\ref{eq:T-matrix definition})
can be used to obtain a $T$-matrix of a rotated or mirror-reflected
particle. Let $T$ be the $T$-matrix of an original particle; the
$T$-matrix of a particle physically transformed by operation $g\in O(3)$
is then 
\begin{equation}
T'_{\tau lm;\tau'l'm'}=\sum_{\mu=-l}^{l}\sum_{\mu'=-l'}^{l'}\left(D_{\mu,m}^{\tau l}\left(g\right)\right)^{*}T_{\tau l\mu;\tau'l'm'}D_{m',\mu'}^{\tau l}\left(g\right).\label{eq:T-matrix of a transformed particle}
\end{equation}
If the particle is symmetric (so that $g$ produces a particle indistinguishable
from the original one), the $T$-matrix must remain invariant under
the transformation (\ref{eq:T-matrix of a transformed particle}),
$T'_{\tau lm;\tau'l'm'}=T{}_{\tau lm;\tau'l'm'}$. Explicit forms
of these invariance properties for the most imporant point group symmetries
can be found in \cite{schulz_point-group_1999}.

If the field expansion is done around a point $\vect r_{p}$ different
from the global origin, as in \ref{eq:E field expansion multiparticle},
we have\foreignlanguage{english}{
\begin{align}
\vect E\left(\omega,R_{g}\vect r\right) & =\sum_{\tau=1,2}\sum_{l=1}^{\infty}\sum_{m=-l}^{+l}\sum_{m'=-l}^{l}\left(\rcoeffptlm p{\tau}lmD_{m',\mu'}^{\tau l}\left(g\right)\vswfrtlm{\tau}l{m'}\left(k\left(\vect r-R_{g}\vect r_{p}\right)\right)+\outcoeffptlm p{\tau}lmD_{m',\mu'}^{\tau l}\left(g\right)\vswfouttlm{\tau}l{m'}\left(k\left(\vect r-R_{g}\vect r_{p}\right)\right)\right).\label{eq:rotated E field expansion around outside origin}
\end{align}
}

\begin{figure}
\caption{Scatterer orbits under $D_{2}$ symmetry. Particles $A,B,C,D$ lie
outside of origin or any mirror planes, and together constitute an
orbit of the size equal to the order of the group, $\left|D_{2}\right|=4$.
Particles $E,F$ lie on the $xz$ plane, hence the corresponding reflection
maps each of them to itself, but the $yz$ reflection (or the $\pi$
rotation around the $z$ axis) maps them to each other, forming a
particle orbit of size 2. The particle $O$ in the very origin is
always mapped to itself, constituting its own orbit.}\label{fig:D2-symmetric structure particle orbits}
\end{figure}

With these transformation properties in hand, we can proceed to the
effects of point symmetries on the whole many-particle system. Let
us have a many-particle system symmetric with respect to a point group
$G$. A symmetry operation $g\in G$ determines a permutation of the
particles: $p\mapsto\pi_{g}(p)$, $p\in\mathcal{P}$. For a given
particle $p$, we will call the set of particles onto which any of
the symmetries maps the particle $p$, i.e. the set $\left\{ \pi_{g}\left(p\right);g\in G\right\} $,
as the \emph{orbit} of particle $p$. The whole set $\mathcal{P}$
can therefore be divided into the different particle orbits; an example
is in Fig. \ref{fig:D2-symmetric structure particle orbits}. The
importance of the particle orbits stems from the following: in the
multiple-scattering problem, outside of the scatterers  one has \foreignlanguage{english}{
\begin{align}
\vect E\left(\omega,R_{g}\vect r\right) & =\sum_{\tau=1,2}\sum_{l=1}^{\infty}\sum_{m=-l}^{+l}\sum_{m'=-l}^{l}\left(\rcoeffptlm p{\tau}lmD_{m,\mu'}^{\tau l}\left(g\right)\vswfrtlm{\tau}l{m'}\left(k\left(\vect r-\vect r_{\pi_{g}(p)}\right)\right)+\outcoeffptlm p{\tau}lmD_{m,\mu'}^{\tau l}\left(g\right)\vswfouttlm{\tau}l{m'}\left(k\left(\vect r-\vect r_{p}\right)\right)\right)\label{eq:rotated E field expansion around outside origin-1}\\
 & =\sum_{\tau=1,2}\sum_{l=1}^{\infty}\sum_{m=-l}^{+l}\sum_{m'=-l}^{l}\left(\rcoeffptlm{\pi_{g}^{-1}(p)}{\tau}lmD_{m,\mu'}^{\tau l}\left(g\right)\vswfrtlm{\tau}l{m'}\left(k\left(\vect r-\vect r_{p}\right)\right)+\outcoeffptlm{\pi_{g}^{-1}(p)}{\tau}lmD_{m,\mu'}^{\tau l}\left(g\right)\vswfouttlm{\tau}l{m'}\left(k\left(\vect r-\vect r_{p}\right)\right)\right).
\end{align}
This means that the field expansion coefficients $\rcoeffp p,\outcoeffp p$
transform as 
\begin{align}
\rcoeffptlm p{\tau}lm & \mapsto\rcoeffptlm{\pi_{g}^{-1}(p)}{\tau}lmD_{m,\mu'}^{\tau l}\left(g\right),\nonumber \\
\outcoeffptlm p{\tau}lm & \mapsto\outcoeffptlm{\pi_{g}^{-1}(p)}{\tau}lmD_{m,\mu'}^{\tau l}\left(g\right).\label{eq:excitation coefficient under symmetry operation}
\end{align}
Obviously, the expansion coefficients belonging to particles in different
orbits do not mix together. As before, we introduce a short-hand block-matrix
notation for \ref{eq:excitation coefficient under symmetry operation}}

\selectlanguage{english}%
\begin{align}
\rcoeff & \mapsto D\left(g\right)a,\nonumber \\
\outcoeff & \mapsto D\left(g\right)\outcoeff.\label{eq:excitation coefficient under symmetry operation block form}
\end{align}

\selectlanguage{finnish}%

\subsection{Irrep decomposition}

\subsection{Periodic systems}

\selectlanguage{english}%

